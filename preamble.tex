% File name: preamble.tex
% Author: Shavak Sinanan <shavak@gmail.com>
% Description: Preamble for mathematical articles.

% Packages
\usepackage[round,sort]{natbib}
\bibliographystyle{authordate1}
\usepackage{enumitem}
\usepackage{multirow}
\usepackage{longtable}
\usepackage{amsthm}
\usepackage{amsmath}% American Mathematical Society macros - essential!
\usepackage{amssymb}% contains amsfonts
\usepackage{mathtools}
\usepackage{mathrsfs}
\usepackage{commath}
\usepackage{isomath}
\usepackage{upgreek}
\usepackage{mathabx}
\usepackage{xfrac}
\usepackage{accents}
\usepackage[integrals]{wasysym}
\usepackage{url}
\usepackage{graphicx}
\usepackage[all]{xy}
\usepackage[nodayofweek]{datetime}% change the format of printed dates (no american style!)
\usepackage[obeyFinal]{todonotes}

\usepackage[pdftex,colorlinks]{hyperref}% creates hypertext links in pdf files (natbib compatible)

% Date and Time format
\newdateformat{yyyydate}{\THEYEAR}
\newdateformat{yydate}{\THEYEAR}

% Operators
\DeclareMathOperator{\Sym}{Sym}% symmetric group
\DeclareMathOperator{\Alt}{Alt}% alternating group
\DeclareMathOperator{\Aut}{Aut}% automorphism group
\DeclareMathOperator{\End}{End}% endomorphism group
\DeclareMathOperator{\Hom}{Hom}% homomorphism group
\DeclareMathOperator{\Ran}{Ran}% range
\DeclareMathOperator{\Dom}{Dom}% domain
\DeclareMathOperator{\Codom}{Codom}% codomain
\DeclareMathOperator{\Mat}{Mat}
\DeclareMathOperator{\tr}{tr}% trace
\DeclareMathOperator{\GL}{GL}% general linear group
\DeclareMathOperator{\SL}{SL}% special linear group
\DeclareMathOperator{\PGL}{PGL}% projective general linear group
\DeclareMathOperator{\PSL}{PSL}% projective special linear group
\DeclareMathOperator{\GU}{GU}% general unitary group
\DeclareMathOperator{\SU}{SU}% special unitary group
\DeclareMathOperator{\PGU}{PGU}% projective general unitary group
\DeclareMathOperator{\PSU}{PSU}% projective special unitary group
\DeclareMathOperator{\AGL}{AGL}% affine general linear group
\DeclareMathOperator{\Id}{Id}
\DeclareMathOperator{\Grp}{Grp}
\DeclareMathOperator{\Syl}{Syl}
\DeclareMathOperator{\Int}{Int}
\DeclareMathOperator{\Ext}{Ext}
\DeclareMathOperator{\Bd}{Bd}
\DeclareMathOperator{\Unif}{Unif}
\DeclareMathOperator{\Bern}{Bern}
\DeclareMathOperator{\Bin}{Bin}
\DeclareMathOperator{\prob}{\mathbf{P}}
\DeclareMathOperator{\expect}{\mathbf{E}}
\DeclareMathOperator{\Var}{\mathbf{Var}}
\DeclareMathOperator{\Cov}{\mathbf{Cov}}
\DeclareMathOperator{\rank}{rank}
\DeclareMathOperator{\nullity}{nullity}
\DeclareMathOperator{\diag}{diag}
\DeclareMathOperator{\supp}{supp}
\DeclareMathOperator{\fix}{fix}
\DeclareMathOperator{\grad}{grad}
\DeclareMathOperator{\dep}{dep}
\DeclareMathOperator{\lcm}{lcm}
\DeclareMathOperator{\diam}{diam}
\DeclareMathOperator{\vol}{vol}

% Symbols and constants
\newcommand{\nat}{\ensuremath{\mathbb{N}}}% natural numbers
\newcommand{\zed}{\ensuremath{\mathbb{Z}}}% integers
\newcommand{\rat}{\ensuremath{\mathbb{Q}}}% rationals
\newcommand{\real}{\ensuremath{\mathbb{R}}}% real numbers
\newcommand{\complex}{\ensuremath{\mathbb{C}}}% complex numbers
\newcommand{\field}{\ensuremath{\mathbb{F}}}% field
\newcommand{\indicator}{\ensuremath{\mathbb{I}}}% indicator
\newcommand{\rie}{\ensuremath{\mathscr{R}}}% Riemann integrable
\newcommand{\circlegroup}{\ensuremath{\mathbb{T}}}% unit circle
\newcommand{\jacobian}{\ensuremath{\mathbf{J}}}
\newcommand{\hessian}{\ensuremath{\mathbf{H}}}
\newcommand{\lowint}{\underline{\int}}
\newcommand{\upint}{\overline{\int}}
\newcommand{\mathup}[1]{\ensuremath{\mathrm{#1}}}
\newcommand{\ramuno}{\ensuremath{\mathup{i}}}
\newcommand{\euler}{\ensuremath{\mathup{e}}}
\newcommand{\cpi}{\ensuremath{\uppi}}
\newcommand{\given}{\ensuremath{\left.\middle|\right.}}
\renewcommand{\vec}{\vectorsym}
\newcommand{\dotprod}{\cdotp}

% Hyphenation fixes
\newcommand{\letdash}[1]{$#1$\nobreakdash-\hspace{0pt}}% for n-element, k-transitive etc
\newcommand{\numdash}{\nobreakdash--}

% Theorems
\theoremstyle{plain}
\newtheorem{theorem}{Theorem}[section]
\newtheorem{lemma}[theorem]{Lemma}
\newtheorem{corollary}[theorem]{Corollary}
\newtheorem{proposition}[theorem]{Proposition}
\newtheorem{observation}[theorem]{Observation}
\newtheorem*{rank-nullity}{The Rank--Nullity Theorem}
\newtheorem*{cayley-hamilton}{The Cayley--Hamilton Theorem}
\newtheorem*{cauchy-schwarz}{The Cauchy--Schwarz Inequality}
\newtheorem*{cramer}{Cramer's Rule}
\newtheorem*{nested-interval}{The Nested Interval Theorem}
\newtheorem*{heine-borel}{The Heine--Borel Theorem}
\newtheorem*{brouwer-fixed-point}{Brouwer's Fixed Point Theorem}
\newtheorem*{bolzano-weierstrass}{The Bolzano--Weierstrass Theorem}
\newtheorem*{monotone-sequence}{The Monotone Sequence Theorem}
\newtheorem*{monotone-convergence}{The Monotone Convergence Theorem}
\newtheorem*{l-hopital}{L'H\^opital's Rule}
\newtheorem*{comparison-test}{The Comparison Test}
\newtheorem*{condensation-test}{Cauchy's Condensation Test}
\newtheorem*{root-test}{The Root Test}
\newtheorem*{ratio-test}{d'Alembert's Ratio Test}
\newtheorem*{integral-test}{The Integral Test}
\newtheorem*{weierstrass-m}{Weierstrass M-test}
\newtheorem*{abel-sum}{Abel's Theorem}
\newtheorem*{evt}{The Extreme-Value Theorem}
\newtheorem*{epsilon-neighbourhood}{The $\epsilon$-Neighbourhood Theorem}
\newtheorem*{ivt}{The Intermediate-Value Theorem}
\newtheorem*{gmvt}{The Generalised Mean-Value Theorem}
\newtheorem*{mvt}{The Mean-Value Theorem}
\newtheorem*{rolle}{Rolle's Theorem}
\newtheorem*{ftoc}{The Fundamental Theorem of Calculus}
\newtheorem*{chain-rule}{The Chain Rule}
\newtheorem*{inverse-function}{The Inverse Function Theorem}
\newtheorem*{caratheodory-extension}{The Carath\'{e}odory Extension Theorem}
\newtheorem*{fubini}{Fubini's Theorem}
\newtheorem*{change-of-variable}{Change of Variable}
\newtheorem*{change-of-variables}{Change of Variables}
\newtheorem*{integ-by-parts}{Integration by Parts}
\newtheorem*{taylor}{Taylor's Theorem}
\newtheorem*{total-probability}{Law of Total Probability}
\newtheorem*{total-expectation}{Law of Total Expectation}
\newtheorem*{total-variance}{Law of Total Variance}
\newtheorem*{bayes}{Bayes' Theorem}
\newtheorem*{probability-integral-transform}{The Probability Integral Transform}
\newtheorem*{wna}{Weak No-arbitrage Condition}
\newtheorem*{sna}{Strong No-arbitrage Condition}
\newtheorem*{first-derivative-test}{The First Derivative Test}
\newtheorem*{second-derivative-test}{The Second Derivative Test}
\newtheorem*{chebychev}{Chebychev's Inequality}
\newtheorem*{law-of-averages}{The Law of Averages}
\newtheorem*{weak-lln}{The Weak Law of Large Numbers}
\newtheorem*{strong-lln}{The Strong Law of Large Numbers}
\newtheorem*{levy-continuity}{L\'{e}vy's Continuity Theorem}
\newtheorem*{central-limit}{The Central Limit Theorem}

% Definition style.
\theoremstyle{definition}
\newtheorem{definition}[theorem]{Definition}
\newtheorem{xca}[theorem]{Exercise}
\newtheorem{problem}[theorem]{Problem}
\newtheorem*{problem*}{Problem}

% Remark style (roman body text with no added vertical space above or below)
\theoremstyle{remark}
\newtheorem{rmk}[theorem]{Remark}
\newtheorem{eg}[theorem]{Example}

\newenvironment{soln}{\begin{proof}[Solution]}{\end{proof}}

\numberwithin{equation}{section}

\documentclass[a4paper,reqno,12pt]{amsart}

% File name: preamble.tex
% Author: Shavak Sinanan <shavak@gmail.com>
% Description: Preamble for mathematical articles.

% Packages
\usepackage[round,sort]{natbib}
\bibliographystyle{authordate1}
\usepackage{enumitem}
\usepackage{multirow}
\usepackage{longtable}
\usepackage{amsthm}
\usepackage{amsmath}% American Mathematical Society macros - essential!
\usepackage{amssymb}% contains amsfonts
\usepackage{mathtools}
\usepackage{mathrsfs}
\usepackage{commath}
\usepackage{isomath}
\usepackage{upgreek}
\usepackage{mathabx}
\usepackage{xfrac}
\usepackage{accents}
\usepackage[integrals]{wasysym}
\usepackage{url}
\usepackage{graphicx}
\usepackage[all]{xy}
\usepackage[nodayofweek]{datetime}% change the format of printed dates (no american style!)
\usepackage[obeyFinal]{todonotes}

\usepackage[pdftex,colorlinks]{hyperref}% creates hypertext links in pdf files (natbib compatible)

% Date and Time format
\newdateformat{yyyydate}{\THEYEAR}
\newdateformat{yydate}{\THEYEAR}

% Operators
\DeclareMathOperator{\Sym}{Sym}% symmetric group
\DeclareMathOperator{\Alt}{Alt}% alternating group
\DeclareMathOperator{\Aut}{Aut}% automorphism group
\DeclareMathOperator{\End}{End}% endomorphism group
\DeclareMathOperator{\Hom}{Hom}% homomorphism group
\DeclareMathOperator{\Ran}{Ran}% range
\DeclareMathOperator{\Dom}{Dom}% domain
\DeclareMathOperator{\Codom}{Codom}% codomain
\DeclareMathOperator{\Mat}{Mat}
\DeclareMathOperator{\tr}{tr}% trace
\DeclareMathOperator{\GL}{GL}% general linear group
\DeclareMathOperator{\SL}{SL}% special linear group
\DeclareMathOperator{\PGL}{PGL}% projective general linear group
\DeclareMathOperator{\PSL}{PSL}% projective special linear group
\DeclareMathOperator{\GU}{GU}% general unitary group
\DeclareMathOperator{\SU}{SU}% special unitary group
\DeclareMathOperator{\PGU}{PGU}% projective general unitary group
\DeclareMathOperator{\PSU}{PSU}% projective special unitary group
\DeclareMathOperator{\AGL}{AGL}% affine general linear group
\DeclareMathOperator{\Id}{Id}
\DeclareMathOperator{\Grp}{Grp}
\DeclareMathOperator{\Syl}{Syl}
\DeclareMathOperator{\Int}{Int}
\DeclareMathOperator{\Ext}{Ext}
\DeclareMathOperator{\Bd}{Bd}
\DeclareMathOperator{\Unif}{Unif}
\DeclareMathOperator{\Bern}{Bern}
\DeclareMathOperator{\Bin}{Bin}
\DeclareMathOperator{\prob}{\mathbf{P}}
\DeclareMathOperator{\expect}{\mathbf{E}}
\DeclareMathOperator{\Var}{\mathbf{Var}}
\DeclareMathOperator{\Cov}{\mathbf{Cov}}
\DeclareMathOperator{\rank}{rank}
\DeclareMathOperator{\nullity}{nullity}
\DeclareMathOperator{\diag}{diag}
\DeclareMathOperator{\supp}{supp}
\DeclareMathOperator{\fix}{fix}
\DeclareMathOperator{\grad}{grad}
\DeclareMathOperator{\dep}{dep}
\DeclareMathOperator{\lcm}{lcm}
\DeclareMathOperator{\diam}{diam}
\DeclareMathOperator{\vol}{vol}

% Symbols and constants
\newcommand{\nat}{\ensuremath{\mathbb{N}}}% natural numbers
\newcommand{\zed}{\ensuremath{\mathbb{Z}}}% integers
\newcommand{\rat}{\ensuremath{\mathbb{Q}}}% rationals
\newcommand{\real}{\ensuremath{\mathbb{R}}}% real numbers
\newcommand{\complex}{\ensuremath{\mathbb{C}}}% complex numbers
\newcommand{\field}{\ensuremath{\mathbb{F}}}% field
\newcommand{\indicator}{\ensuremath{\mathbb{I}}}% indicator
\newcommand{\rie}{\ensuremath{\mathscr{R}}}% Riemann integrable
\newcommand{\circlegroup}{\ensuremath{\mathbb{T}}}% unit circle
\newcommand{\jacobian}{\ensuremath{\mathbf{J}}}
\newcommand{\hessian}{\ensuremath{\mathbf{H}}}
\newcommand{\lowint}{\underline{\int}}
\newcommand{\upint}{\overline{\int}}
\newcommand{\mathup}[1]{\ensuremath{\mathrm{#1}}}
\newcommand{\ramuno}{\ensuremath{\mathup{i}}}
\newcommand{\euler}{\ensuremath{\mathup{e}}}
\newcommand{\cpi}{\ensuremath{\uppi}}
\newcommand{\given}{\ensuremath{\left.\middle|\right.}}
\renewcommand{\vec}{\vectorsym}
\newcommand{\dotprod}{\cdotp}

% Hyphenation fixes
\newcommand{\letdash}[1]{$#1$\nobreakdash-\hspace{0pt}}% for n-element, k-transitive etc
\newcommand{\numdash}{\nobreakdash--}

% Theorems
\theoremstyle{plain}
\newtheorem{theorem}{Theorem}[section]
\newtheorem{lemma}[theorem]{Lemma}
\newtheorem{corollary}[theorem]{Corollary}
\newtheorem{proposition}[theorem]{Proposition}
\newtheorem{observation}[theorem]{Observation}
\newtheorem*{rank-nullity}{The Rank--Nullity Theorem}
\newtheorem*{cayley-hamilton}{The Cayley--Hamilton Theorem}
\newtheorem*{cauchy-schwarz}{The Cauchy--Schwarz Inequality}
\newtheorem*{cramer}{Cramer's Rule}
\newtheorem*{nested-interval}{The Nested Interval Theorem}
\newtheorem*{heine-borel}{The Heine--Borel Theorem}
\newtheorem*{brouwer-fixed-point}{Brouwer's Fixed Point Theorem}
\newtheorem*{bolzano-weierstrass}{The Bolzano--Weierstrass Theorem}
\newtheorem*{monotone-sequence}{The Monotone Sequence Theorem}
\newtheorem*{monotone-convergence}{The Monotone Convergence Theorem}
\newtheorem*{l-hopital}{L'H\^opital's Rule}
\newtheorem*{comparison-test}{The Comparison Test}
\newtheorem*{condensation-test}{Cauchy's Condensation Test}
\newtheorem*{root-test}{The Root Test}
\newtheorem*{ratio-test}{d'Alembert's Ratio Test}
\newtheorem*{integral-test}{The Integral Test}
\newtheorem*{weierstrass-m}{Weierstrass M-test}
\newtheorem*{abel-sum}{Abel's Theorem}
\newtheorem*{evt}{The Extreme-Value Theorem}
\newtheorem*{epsilon-neighbourhood}{The $\epsilon$-Neighbourhood Theorem}
\newtheorem*{ivt}{The Intermediate-Value Theorem}
\newtheorem*{gmvt}{The Generalised Mean-Value Theorem}
\newtheorem*{mvt}{The Mean-Value Theorem}
\newtheorem*{rolle}{Rolle's Theorem}
\newtheorem*{ftoc}{The Fundamental Theorem of Calculus}
\newtheorem*{chain-rule}{The Chain Rule}
\newtheorem*{inverse-function}{The Inverse Function Theorem}
\newtheorem*{caratheodory-extension}{The Carath\'{e}odory Extension Theorem}
\newtheorem*{fubini}{Fubini's Theorem}
\newtheorem*{change-of-variable}{Change of Variable}
\newtheorem*{change-of-variables}{Change of Variables}
\newtheorem*{integ-by-parts}{Integration by Parts}
\newtheorem*{taylor}{Taylor's Theorem}
\newtheorem*{total-probability}{Law of Total Probability}
\newtheorem*{total-expectation}{Law of Total Expectation}
\newtheorem*{total-variance}{Law of Total Variance}
\newtheorem*{bayes}{Bayes' Theorem}
\newtheorem*{probability-integral-transform}{The Probability Integral Transform}
\newtheorem*{wna}{Weak No-arbitrage Condition}
\newtheorem*{sna}{Strong No-arbitrage Condition}
\newtheorem*{first-derivative-test}{The First Derivative Test}
\newtheorem*{second-derivative-test}{The Second Derivative Test}
\newtheorem*{chebychev}{Chebychev's Inequality}
\newtheorem*{law-of-averages}{The Law of Averages}
\newtheorem*{weak-lln}{The Weak Law of Large Numbers}
\newtheorem*{strong-lln}{The Strong Law of Large Numbers}
\newtheorem*{levy-continuity}{L\'{e}vy's Continuity Theorem}
\newtheorem*{central-limit}{The Central Limit Theorem}

% Definition style.
\theoremstyle{definition}
\newtheorem{definition}[theorem]{Definition}
\newtheorem{xca}[theorem]{Exercise}
\newtheorem{problem}[theorem]{Problem}
\newtheorem*{problem*}{Problem}

% Remark style (roman body text with no added vertical space above or below)
\theoremstyle{remark}
\newtheorem{rmk}[theorem]{Remark}
\newtheorem{eg}[theorem]{Example}

\newenvironment{soln}{\begin{proof}[Solution]}{\end{proof}}

\numberwithin{equation}{section}


\begin{document}
	
\title{Solution to the ``Christmas Eve'' Riddle}
\author{Shavak Sinanan}
	
\maketitle

\begin{problem*}
    Let \(n\) be a positive integer such that \(n + 1\) is divisible by \(24\). Show that then also the sum of the divisors of \(n\) is divisible by \(24\).
\end{problem*}

\begin{soln}
    First, more generally, recall that if a positive integer \(n\) has unique prime factorisation
    \[
    p_1^{\alpha_1} p_2^{\alpha_2} \dotsb p_k^{\alpha_k}\text{,}
    \]
    then the number of divisors of \(n\) is given by
    \[
    (\alpha_1 + 1)(\alpha_2 + 1) \dotsb (\alpha_k + 1)\text{.}
    \]
    
    To see this, observe that any divisor of \(n\) must have the form
    \[
    p_1^{\beta_1} p_2^{\beta_2} \dotsb p_k^{\beta_k}\text{,}
    \]
    where \(0 \leq \beta_j \leq \alpha_j\) for each \(j\). There are therefore \(\alpha_j + 1\) choices for the value of \(\beta_j\), with each different tuple \((\beta_1, \beta_2, \dotsc, \beta_k)\) mapping to a different divisor of \(n\).
    
    It follows immediately that
    \begin{enumerate}[label=\textbf{Fact} \Roman*]
	\item \label{itm:even_number_divisors} \emph{The number of divisors of a positive integer \(n\) is odd if and only if \(n\) is a perfect square}.
    \end{enumerate}
    
    Note also the following, both of which are trivially checked.
    \begin{enumerate}[resume,label=\textbf{Fact} \Roman*]
	\item \label{itm:quadratic_residues_mod3} \emph{The square of an integer can only be congruent to \(0\) or \(1\) modulo \(3\).}\footnote{An integer which appears as a remainder of some square integer modulo a positive integer \(x\) is called a \emph{quadratic residue} of \(x\).}
	\item \label{itm:quadratic_residues_mod8} \emph{The square of an integer can only be congruent to \(0\), \(1\), or \(4\) modulo \(8\).}
    \end{enumerate}
    
    Now suppose that \(n + 1\) is divisible by \(24\). One has \(n \equiv 2 \pmod 3\), and so \(n\) cannot be a perfect square by \ref{itm:quadratic_residues_mod3}, and must have an even number of divisors by \ref{itm:even_number_divisors}.
    
    Let \(1 = d_1 < d_2 < \dotsb < d_m\) be the divisors of \(n\) which do not exceed \(\sqrt{n}\). The divisors of \(n\) can be paired-off and listed as
    \[
    d_1, \frac{n}{d_1}, d_2, \frac{n}{d_2},\dotsc, d_m, \frac{n}{d_m}\text{,}  
    \]
    and their sum may be written as
    \[
    \sum_{j = 1}^m \left(d_j + \frac{n}{d_j}\right)\text{.}
    \]
    It is sufficient to prove that each term in the sum above is divisible by \(24\).
    
    Let \(d\) be an arbitrary divisor of \(n\) and consider the expression
    \[
    d + \frac{n}{d} = \frac{d^2 + n}{d}\text{.}
    \]
    Since \(n + 1\) is divisible by \(24\), one must have that \(n\) is co-prime to \(24\), whence so are any of its divisors.
    
    Thus, the problem is reduced to proving that the numerator of the expression above is divisible by \(24\).
    
    By \ref{itm:quadratic_residues_mod3},  \(d^2\) must be congruent to \(0\) or \(1\) modulo \(3\). The former implies that \(d\) is divisible by \(3\), which is impossible since \(d\) is co-prime to \(24\). Hence, \(d^2 \equiv 1 \pmod{3}\).
    
    By \ref{itm:quadratic_residues_mod8}, \(d^2\) must be congruent to \(0\), \(1\), or \(4\) modulo \(8\). Either of \(d^2 \equiv 0 \pmod{8}\) or \(d^2 \equiv 4 \pmod{8}\) implies that \(d\) is even, which is impossible since \(d\) is co-prime to \(24\). Hence, \(d^2 \equiv 1 \pmod{8}\).
    
    This shows that \(d^2 \equiv 1 \pmod{24}\), from which it follows easily that
    \[
    d^2 + n = (d^2 - 1) + (n + 1)
    \]
    is divisible by \(24\), as required.
\end{soln}

\end{document}
